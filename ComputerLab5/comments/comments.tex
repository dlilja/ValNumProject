\documentclass[a4paper, 11 pt]{amsart}
\usepackage{amsfonts}
\usepackage{amssymb}
\usepackage{graphicx}
\usepackage{hyperref}
\usepackage{aliascnt}
\usepackage{mathtools}
\usepackage{geometry}
\usepackage{fancyhdr}

\pagestyle{fancy}
\lhead{Dan Lilja}
\rhead{Validated Numerics \\
	Computer lab 5}

\begin{document}

\thispagestyle{fancy}

\hfill

\section*{Further comments}

The period function is computed by starting at a point $(x,0)$ and integrating
to a point $(0,y)$. The period is then given by 4 times this integration time.

For doubles, a Taylor method is used for integrating and a Newton method is used
for finding the correct integration time $T$ giving $x(T) = 0$ using an initial
guess provided by the user.

For intervals, a rigorous Taylor method is used for integrating and a Krawczyk
method is used for finding an interval enclosure of the time $T$ such that
$x(T)=0$.

In both cases, a user provided tolerance is used to determine how close the
solution time needs to be the actual time.

\end{document}