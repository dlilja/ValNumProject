\documentclass[a4paper, 11 pt]{amsart}
\usepackage{amsfonts}
\usepackage{amssymb}
\usepackage{graphicx}
\usepackage{hyperref}
\usepackage{aliascnt}
\usepackage{mathtools}
\usepackage{geometry}
\usepackage{fancyhdr}
\usepackage{euscript}
\usepackage{tikz,tkz-euclide,tikz-cd}
\usepackage{float}
\usepackage{caption}
\usepackage{subcaption}
\usepackage{csquotes}
\usetkzobj{all}

%Proper open/closed sub/supset-notation
\newcommand{\osub}{\overset{\scriptscriptstyle{\mathrm{open}}}{\subset}}
\newcommand{\csub}{\overset{\scriptscriptstyle{\mathrm{closed}}}{\subset}}
\newcommand{\osup}{\overset{\scriptscriptstyle{\mathrm{open}}}{\supset}}
\newcommand{\csup}{\overset{\scriptscriptstyle{\mathrm{closed}}}{\supset}}
\newcommand{\cpsub}{\overset{\scriptscriptstyle{\mathrm{compact}}}{\subset}}
\newcommand{\cpsup}{\overset{\scriptscriptstyle{\mathrm{compact}}}{\supset}}

%Vector styles -begin-
\renewcommand{\v}[1]{\boldsymbol{#1}}
\newcommand{\vf}[1]{\boldsymbol{\bar{#1}}}
\newcommand{\ve}[1]{\hat{\boldsymbol{#1}}}
\newcommand{\vb}[1]{\underline{\boldsymbol{#1}}}
%Vector styles -end-

%Custom made symbols (proper ones -begin-
\newcommand*{\defeq}{\mathrel{\vcenter{\baselineskip0.65ex \lineskiplimit0pt
			\hbox{$\raisebox{-0.200ex}{\scriptsize.}$}\hbox{\scriptsize.}}}
	=} %Proper :=
\newcommand*{\eqdef}{=\mathrel{\vcenter{\baselineskip0.45ex \lineskiplimit0pt
			\hbox{$\raisebox{0.10ex}{\scriptsize.}$}\hbox{\scriptsize.}}}
} %Proper =:
\newcommand{\co}{\colon\thinspace} %Proper colon
\newcommand{\st}{\vert\thinspace} %Proper "such that"
\newcommand{\sca}{\raisebox{.3ex}{\tiny$\ \bullet\ $}} %Proper scalar product

%Custom made symbols (proper ones) -end-

%Misc symbols -begin-
\renewcommand{\d}{\partial}
\newcommand{\del}{\nabla}
\newcommand{\R}{\mathbb{R}}
\newcommand{\N}{\mathbb{N}}
\renewcommand{\P}{\mathbb{P}}
\newcommand{\Z}{\mathbb{Z}}
\newcommand{\Q}{\mathbb{Q}}
\newcommand{\C}{\mathbb{C}}
\newcommand{\e}{\varepsilon}
\newcommand{\f}{\varphi}
\newcommand{\E}{\mathbb{E}}
\newcommand{\RP}{\mathbb{R}\mathrm{P}}
%Misc symbols -end-

%Custom defined math operators -begin-
\DeclareMathOperator{\sgn}{sgn}
\DeclareMathOperator{\lk}{lk}
\DeclareMathOperator{\im}{im}
\DeclareMathOperator{\id}{id}
\DeclareMathOperator{\ind}{ind}
\DeclareMathOperator{\codim}{codim}
%Custom defined math operators -end-

%Parenthesis and arrows -begin-
\newcommand{\ekviv}{\ \Leftrightarrow \ }
\newcommand{\imp}{\ \Rightarrow \ }
\newcommand{\paren}[1]{\left( #1 \right)}
\newcommand{\parenb}[1]{\left[ #1 \right]}
\newcommand{\parenm}[1]{\left\{ #1 \right\}}
\newcommand{\ip}[2]{\left\langle #1, #2 \right\rangle}
\newcommand{\absv}[1]{\left\vert #1 \right\vert}
\newcommand{\norm}[1]{\left\Vert #1 \right\Vert}
\newcommand{\hak}[1]{\left \langle #1 \right \rangle}
\newcommand{\slfrac}[2]{\left. #1 \middle / #2 \right .}
%Parenthesis and arrows -end-

\newcommand{\rad}[2]{#1_{#2\raisebox{.1ex}{\tiny$ \bullet$}}} %Row of a matrix notation
\newcommand{\kol}[2]{#1_{\raisebox{.1ex}{\tiny$ \bullet$} #2}} %Column of a matrix notation

\newtheorem{theorem}{Theorem}

\newaliascnt{lemma}{theorem}
\newtheorem{lemma}[lemma]{Lemma}
\aliascntresetthe{lemma}
\providecommand{\lemmaautorefname}{Lemma}

\newaliascnt{proposition}{theorem}
\newtheorem{proposition}[proposition]{Proposition}
\aliascntresetthe{proposition}
\providecommand{\propositionautorefname}{Proposition}

\newaliascnt{corollary}{theorem}
\newtheorem{corollary}[corollary]{Corollary}
\aliascntresetthe{corollary}
\providecommand{\corollaryautorefname}{Corollary}

\theoremstyle{definition}

\newaliascnt{definition}{theorem}
\newtheorem{definition}[definition]{Definition}
\aliascntresetthe{definition}
\providecommand{\definitionautorefname}{Definition}

\newaliascnt{example}{theorem}
\newtheorem{example}[example]{Example}
\aliascntresetthe{example}
\providecommand{\exampleautorefname}{Example}

\newaliascnt{exercise}{theorem}
\newtheorem{exercise}[exercise]{Exercise}
\aliascntresetthe{exercise}
\providecommand{\exerciseautorefname}{Exercise}

\theoremstyle{remark}

\newaliascnt{remark}{theorem}
\newtheorem{remark}[remark]{Remark}
\aliascntresetthe{remark}
\providecommand{\remarkautorefname}{Remark}

\numberwithin{equation}{section}

\makeindex

\pagestyle{fancy}
\lhead{Dan Lilja}
\rhead{Validated Numerics \\
	Computer lab 1} % course

\begin{document}

\thispagestyle{fancy}

\hfill

\section*{Further comments}

\subsection*{Exercise 5a}

Using monotonicity of the integrand allows us to find rigorous bounds by simply
computing lower and upper Riemann sums.

Since we haven't yet implemented rigorous versions of the standard functions I
use the ones in \texttt{cmath} for now.
\subsection*{Exercise 5b}

In solving this exercise I perform two changes of variables. The first change 
of variables is $ y=x^{2} $, transforming the integral as

\begin{align*}
	\int_{1}^{\infty}\sin(x^{2})x^{-3}dx 
	& = \dfrac{1}{2}\int_{1}^{\infty}\sin(y)y^{-2}dy \, .
\end{align*}
From here I divide the integral into three parts that I call the head, the main 
part and the tail. These are, respectively,
\begin{align*}
	\dfrac{1}{2} & \int_{1}^{\frac{\pi}{2}}\sin(y)y^{-2}dy \, , \\
	\dfrac{1}{2} & \int_{\frac{\pi}{2}}^{256\frac{3\pi}{2}}\sin(y)y^{-2}dy \, , 
	\\
	\dfrac{1}{2} & \int_{256\frac{3\pi}{2}}^{\infty}\sin(y)y^{-2}dy \, .
\end{align*}

The head is calculated using that $ \sin(y) $ is increasing and $ y^{-2} $ 
decreasing by dividing $ \left[1,\frac{\pi}{2}\right] $ into small pieces and 
on each piece approximating the value by multiplying the two functions largest 
values for the upper bound and smallest value for the lower bound.

For the main part I use the same idea and the fact that $ \sin(y) $ alternates 
between increasing and decreasing on the intervals $ \left[ 
\frac{(1+2k)\pi}{2},\frac{(1+2(k+1))\pi}{2} \right] $. I also perform another 
change of variables, $ y=\pi dz $ two remove several occurrences of $ \pi $ in 
the integrand and the interval boundaries.  

Finally, the tail is approximated by simply forgetting the $ \sin $ factor of 
the integrand and using
\begin{align*}
	\dfrac{1}{2\pi}\int_{256\frac{3}{2}}^{\infty}z^{-2}dz & = 
	\dfrac{1}{2\pi}\left[ -z^{-1} \right]_{256\frac{3}{2}}^{\infty} \\
	& = \dfrac{1}{2\pi\cdot 3\cdot 128} \, .
\end{align*}

Regarding the sharpness of the achieved bounds, just as my solution of exercise 
5a it depends on the standard implementations of \texttt{sin} and 
\texttt{pow} respecting the roundings and there is inherent non-sharpness in 
the way I handle the value of $ \pi $.

\appendix

%    Bibliographies can be prepared with BibTeX using amsplain,
%    amsalpha, or (for "historical" overviews) natbib style.

%\begin{bibdiv}
%	\begin{biblist}
%	
%	\end{biblist}
%\end{bibdiv}


\end{document}
